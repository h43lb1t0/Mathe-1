\documentclass[12pt,a4paper]{article}
\usepackage[utf8]{inputenc}
\usepackage[ngerman]{babel}
\usepackage{amsmath, amssymb}
\usepackage{geometry}
\usepackage{pgfplots}
\usepackage{hyperref}
\usepackage{xcolor}
\usepackage{colortbl}
\geometry{a4paper, left=25mm, right=25mm, top=20mm, bottom=20mm}

\begin{document}

\title{Mathe 1}
\author{Tom Haelbich}
\date{WiSe2024/25}

\maketitle
\begin{center}
    \textbf{Hochschule für Technik und Wirtschaft Berlin} \\
    Angewandte Informatik
\end{center}


\tableofcontents
\newpage

\section{Mengenlehre}

\subsection{Element von (\texorpdfstring{$\in$}{in})}
\textbf{Symbol:} $\in$ \\
\textbf{Erklärung:} Das Symbol $\in$ bedeutet, dass ein Element zu einer Menge gehört. \\
\textbf{Beispiel:} $3 \in \mathbb{N}$, was bedeutet, dass die Zahl 3 ein Element der natürlichen Zahlen ist.

\subsection{Kein Element von (\texorpdfstring{$\notin$}{notin})}
\textbf{Symbol:} $\notin$ \\
\textbf{Erklärung:} Das Symbol $\notin$ bedeutet, dass ein Element nicht zu einer Menge gehört. \\
\textbf{Beispiel:} $-1 \notin \mathbb{N}$, da -1 keine natürliche Zahl ist.

\subsection{Für alle (\texorpdfstring{$\forall$}{forall})}
\textbf{Symbol:} $\forall$ \\
\textbf{Erklärung:} Das Symbol $\forall$ bedeutet \textit{für alle}. \\
\textbf{Beispiel:} $\forall x \in \mathbb{N}, x \geq 0$

\subsection{Es existiert (\texorpdfstring{$\exists$}{exists})}
\textbf{Symbol:} $\exists$ \\
\textbf{Erklärung:} Das Symbol $\exists$ bedeutet \textit{es existiert}. \\
\textbf{Beispiel:} $\exists x \in \mathbb{N}, x = 0$

\subsection{Es existiert genau ein (\texorpdfstring{$\exists !$}{exists!})}
\textbf{Symbol:} $\exists !$ \\
\textbf{Erklärung:} Das Symbol $\exists !$ bedeutet \textit{es existiert genau ein}. \\
\textbf{Beispiel:} $\exists ! x \in \mathbb{N}, x = 0$

\subsection{Teilmenge (\texorpdfstring{$\subset$}{subset})}
\textbf{Symbol:} $\subset$ \\
\textbf{Erklärung:} Eine Menge $A$ ist eine \emph{echte Teilmenge} von $B$, wenn alle Elemente von $A$ auch in $B$ sind, aber $A \neq B$. \texttt{"Boolischer Ausdruck"} \\
\textbf{Beispiel:} $\{1,2\} \subset \{1,2,3\}$

\subsection{Teilmenge oder gleich (\texorpdfstring{$\subseteq$}{subseteq})}
\textbf{Symbol:} $\subseteq$ \\
\textbf{Erklärung:} Eine Menge $A$ ist eine \emph{Teilmenge} von $B$, wenn alle Elemente von $A$ auch in $B$ sind. \texttt{"Boolischer Ausdruck"} \\
\textbf{Beispiel:} $\{1,2,3\} \subseteq \{1,2,3\}$

\subsection{Vereinigung (\texorpdfstring{$\cup$}{cup})}
\textbf{Symbol:} $\cup$ \\
\textbf{Erklärung:} Die Vereinigung zweier Mengen $A$ und $B$ ist die Menge aller Elemente, die in $A$ oder in $B$ sind. \\
\textbf{Beispiel:} $\{1,2\} \cup \{2,3\} = \{1,2,3\}$
\begin{center}
    \begin{tikzpicture}
        \fill[blue!20] (-1,0) circle (1.5); % Färbung für A
    \fill[blue!20] (1,0) circle (1.5); % Färbung für B

    % Zeichne die Ränder der Kreise
    \draw (-1,0) circle (1.5) node[left] {$A$}; % Rand von A
    \draw (1,0) circle (1.5) node[right] {$B$}; % Rand von B

    % Beschriftung der Vereinigungsmenge
    \end{tikzpicture}
\end{center}

\subsection{Durchschnitt (\texorpdfstring{$\cap$}{cap})}
\textbf{Symbol:} $\cap$ \\
\textbf{Erklärung:} Der Durchschnitt zweier Mengen $A$ und $B$ ist die Menge aller Elemente, die sowohl in $A$ als auch in $B$ sind. \\
\textbf{Beispiel:} $\{1,2\} \cap \{2,3\} = \{2\}$
\begin{center}
    \begin{tikzpicture}
        \begin{scope}
            \clip (-1,0) circle (1.5);
            \fill[blue!20] (1,0) circle (1.5);
        \end{scope}
        % Zeichne die Ränder der Kreise
        \draw (-1,0) circle (1.5) node[left] {$A$}; % Rand von A
        \draw (1,0) circle (1.5) node[right] {$B$}; % Rand von B
    \end{tikzpicture}
\end{center}

\subsection{Mengendifferenz (ohne) (\texorpdfstring{$\setminus$}{setminus})}
\textbf{Symbol:} $\setminus$ \\
\textbf{Erklärung:} Die Mengendifferenz $A \setminus B$ ist die Menge aller Elemente, die in $A$, aber nicht in $B$ sind. \\
\textbf{Beispiel:} $\{1,2,3\} \setminus \{2,3\} = \{1\}$
\begin{center}
    \begin{tikzpicture}
        \fill[blue!20] (-1,0) circle (1.5); % Färbung für A
        \begin{scope}
            \clip (1,0) circle (1.5) (-1,0) circle (1.5);
            \fill[white] (1,0) circle (1.5);
        \end{scope}
        % Zeichne die Ränder der Kreise
        \draw (-1,0) circle (1.5) node[left] {$A$}; % Rand von A
        \draw (1,0) circle (1.5) node[right] {$B$}; % Rand von B
    \end{tikzpicture}
\end{center}

\subsection{Potenzmenge (\texorpdfstring{$\mathcal{P}$}{P})}
\textbf{Symbol:} $\mathcal{P}$ \\
\textbf{Erklärung:} Die Potenzmenge $\mathcal{P}(A)$ ist die Menge aller Teilmengen von $A$. \\
\textbf{Beispiel:} Wenn $A = \{1,2\}$, dann ist $\mathcal{P}(A) = \{\emptyset, \{1\}, \{2\}, \{1,2\}\}$

\subsection{Kartesisches Produkt (\texorpdfstring{$\times$}{times})}
\textbf{Symbol:} $\times$ \\
\textbf{Erklärung:} Das kartesische Produkt $A \times B$ ist die Menge aller geordneten Paare $(a,b)$ mit $a \in A$ und $b \in B$. \\
\textbf{Beispiel:} $\{1,2\} \times \{a,b\} = \{(1,a),(1,b),(2,a),(2,b)\}$

\subsection{Cardinalität (\texorpdfstring{$|A|$}{|A|})}
\textbf{Symbol:} $|A|$ \\
\textbf{Erklärung:} Die Anzahl der Elemente in der Menge $A$. \\
\textbf{Beispiel:} Wenn $A = \{1,2,3\}$, dann ist $|A| = 3$

\subsection{Mengennotation}
\begin{itemize}
    \item \textbf{Leeremenge: }$\emptyset$
    \item \textbf{Menge mit Elementen:} $A = \{1,2,3\}$
    \item \textbf{Menge mit Bedingung:} $B = \{x \in \mathbb{N} \mid 1 \leq x \leq 3\} = \{1,2,3\}$
    \item \textbf{Menge als Intervall:} $C = \{x \in \mathbb{R} \mid 1 \leq x < 3\} = [1, 3[$ \\ \textbf{Erklärung:} Klammer offen weg von der Zahl bedeutet, dass die Zahl nicht in der Menge enthalten ist. Alternativ auch eine runde Klammer.
\end{itemize}

\subsection{Wichtige Mengen}

\subsubsection{Natürliche Zahlen (\texorpdfstring{$\mathbb{N}$}{N})}
\textbf{Symbol (Ohne Null):} $\mathbb{N}$ \\
\textbf{Erklärung:} Die Menge der natürlichen Zahlen. \\
\textbf{Beispiel:} $\mathbb{N} = \{1,2,3,\dots\}$ \\
\textbf{Symbol (Mit Null):} $\mathbb{N}_0$ \\
\textbf{Beispiel:} $\mathbb{N}_0 = \{0, 1,2,3,\dots\}$

\subsubsection{Ganze Zahlen (\texorpdfstring{$\mathbb{Z}$}{Z})}
\textbf{Symbol:} $\mathbb{Z}$ \\
\textbf{Erklärung:} Die Menge der ganzen Zahlen. \\
\textbf{Beispiel:} $\mathbb{Z} = \{\dots,-2,-1,0,1,2,\dots\}$

\subsubsection{Reelle Zahlen (\texorpdfstring{$\mathbb{R}$}{R})}
\textbf{Symbol:} $\mathbb{R}$ \\
\textbf{Erklärung:} Die Menge der reellen Zahlen. \\
\textbf{Beispiel:} $\mathbb{R} = \{-5, 0.5, \frac{2}{3}, \cdots\}$.

\section{Logik}

\subsection{Und (\texorpdfstring{$\land$}{land})}
\textbf{Symbol:} $\land$ \\
\textbf{Erklärung:} Das Symbol für die logische Konjunktion (UND). \\
\textbf{Beispiel:} $A \subseteq B \land B \subseteq C$ \\ \\
\textbf{Negation:} $\neg (A \land B) \equiv \neg A \lor \neg B$

\subsection{Oder (\texorpdfstring{$\lor$}{lor})}
\textbf{Symbol:} $\lor$ \\
\textbf{Erklärung:} Das Symbol für die logische Disjunktion (ODER). \\
\textbf{Beispiel:} $A \subseteq B \lor B \subseteq C$ \\ \\
\textbf{Negation:} $\neg (A \lor B) \equiv \neg A \land \neg B$

\subsection{Nicht (\texorpdfstring{$\neg$}{neg})}
\textbf{Symbol:} $\neg$ \\
\textbf{Erklärung:} Das Symbol für die logische Negation (NICHT). \\
\textbf{Beispiel:} $\neg (A \subseteq B)$ \\ \\
\textbf{Negation:} $\neg \neg A \equiv A$

\subsection{Implikation (\texorpdfstring{$\implies$}{implies})}
\textbf{Symbol:} $\implies$ \\
\textbf{Erklärung:} Das Symbol für die logische Implikation (Wenn ... dann). \\
\textbf{Beispiel:} $x > 2 \implies x^2 > 4$ \\ \\
\textbf{Negation:} $\neg (A \implies B) \equiv A \land \neg B$

\subsection{Umkehrimplikation (\texorpdfstring{$\impliedby$}{impliedby})}
\textbf{Symbol:} $\impliedby$ \\
\textbf{Erklärung:} Das Symbol für die logische Umkehrimplikation (Dann ... wenn). \\
\textbf{Beispiel:} $x^2 > 4 \impliedby x > 2$ \\ \\
\textbf{Negation:} $\neg (A \impliedby B) \equiv \neg A \land B$

\subsection{Implikation (\texorpdfstring{$\rightarrow$}{rightarrow})}
\textbf{Symbol:} $\rightarrow$ \\
\textbf{Erklärung:} Ein weiteres Symbol für die logische Implikation. \\
\textbf{Beispiel:} $x > 2 \rightarrow x^2 > 4$ \\ \\
\textbf{Negation:} $\neg (A \rightarrow B) \equiv A \land \neg B$

\subsection{Umkehrimplikation (\texorpdfstring{$\leftarrow$}{leftarrow})}
\textbf{Symbol:} $\leftarrow$ \\
\textbf{Erklärung:} Ein weiteres Symbol für die logische Umkehrimplikation. \\
\textbf{Beispiel:} $x^2 > 4 \leftarrow x > 2$ \\ \\
\textbf{Negation:} $\neg (A \leftarrow B) \equiv \neg A \land B$

\subsection{Äquivalenz (\texorpdfstring{$\iff$}{iff})}
\textbf{Symbol:} $\iff$ \\
\textbf{Erklärung:} Das Symbol für die logische Äquivalenz (genau dann, wenn). \\
\textbf{Beispiel:} $x > 2 \iff x^2 > 4$ \\ \\
\textbf{Negation:} $\neg (A \iff B) \equiv A \iff \neg B$

\subsection{Logische Äquivalenz (\texorpdfstring{$\Leftrightarrow$}{Leftrightarrow})}
\textbf{Symbol:} $\Leftrightarrow$ \\
\textbf{Erklärung:} Ein weiteres Symbol für die logische Äquivalenz. \\
\textbf{Beispiel:} $x \in \mathbb{Q} \Leftrightarrow x$ ist eine rationale Zahl. \\ \\
\textbf{Negation:} $\neg (A \Leftrightarrow B) \equiv A \Leftrightarrow \neg B$

\subsection{Biimplikation (\texorpdfstring{$\leftrightarrow$}{leftrightarrow})}
\textbf{Symbol:} $\leftrightarrow$ \\
\textbf{Erklärung:} Das Symbol für die beidseitige Implikation oder Biimplikation. \\
\textbf{Beispiel:} $P \leftrightarrow Q$ \\ \\
\textbf{Negation:} $\neg (A \leftrightarrow B) \equiv A \leftrightarrow \neg B$

\subsection{Folgerung (\texorpdfstring{$\vdash$}{vdash})}
\textbf{Symbol:} $\vdash$ \\
\textbf{Erklärung:} Das Symbol für die logische Folgerung oder Ableitung. \\
\textbf{Beispiel:} $P \vdash Q$ bedeutet, dass $Q$ aus $P$ folgt.
\section{Zuweisungen}

\subsection{Definition (\texorpdfstring{$:=$}{:=})}
\textbf{Symbol:} $:=$ \\
\textbf{Erklärung:} Das Symbol $:=$ bedeutet \textit{ist definiert als}. \\
\textbf{Beispiel:} $A := \{1,2,3\}$


\section{LGS (Lineare Gleichungssysteme)}
\subsection{Begriffe}
\begin{itemize}
    \item \textbf{Koeffizient:} Die Zahlen vor den Variablen in einer Gleichung.
    \item \textbf{Pivot:} Der Wert in einer Zeile, der als erstes von Null verschieden ist.
    \item \textbf{Pivotvariable:} Die Variable, die in einer Zeile mit dem Pivotwert steht.
    \item \textbf{Pivotzeile:} Die Zeile in einem LGS, in der der Pivotwert steht.
    \item \textbf{Koeffizientenmatrix:} Die Matrix, die aus den Koeffizienten der Variablen besteht.
    \item \textbf{Erweiterte Koeffizientenmatrix:} Die Matrix, die aus den Koeffizienten der Variablen und den Ergebnissen besteht.
\end{itemize}

\subsection{Gauß-Algorithmus}
\textbf{Videoempfehlung:} \href{https://youtu.be/aosbq7Ci7Ec}{Gauß Algorithmus – Lineare Gleichungssysteme lösen | MathemaTrick}
\begin{itemize}
    \item \textbf{Schritt 1:} Stelle die Koeffizientenmatrix auf.
    \item \textbf{Schritt 2:} Führe Zeilenoperationen durch, um Nullen unter den Pivotelementen zu erzeugen.
    \item \textbf{Schritt 3:} Zeilenstuffenform erreichen.
    \begin{itemize}
        \item Unten links von jedem Pivotwert müssen Nullen stehen.
        \item Mit der Linken Seite der Matrix beginnen. Dabei immer die erste Zeile zum verrechnen verwenden.
        \item Bei folgenden spalten immer die Zeile mit 0 links neben dem Pivotwert verwenden.
    \end{itemize}
    \item \textbf{Schritt 4:} Rückwärtseinsetzen, um die Lösung zu finden.
\end{itemize}
\subsection{Mögliche Ergebnisse}
\begin{itemize}
    \item \textbf{Keine Lösung:} Wenn ein Widerspruch entsteht, z.B. $0 = 1$.
    \item \textbf{Unendlich viele Lösungen:} Wenn mehr Unbekannte als Gleichungen vorhanden sind. z.B. Letze Zeile $0 = 0$.
    \item \textbf{Eine Lösung:} Wenn alle Variablen eindeutig bestimmt werden können.
\end{itemize}
\subsection{Beispiel (Eine Lösung)}
\begin{align*}
\textcolor{red}{2x_1} + 3x_2 \textcolor{blue}{-x_3} &= \textcolor{orange}{5} \\
\textcolor{red}{4x_1} - x_2 \textcolor{blue}{+ 5x_3} &= \textcolor{orange}{9} \\
\textcolor{red}{x_1} + 2x_2 \textcolor{blue}{+ 3x_3} &= \textcolor{orange}{6}
\end{align*}
\textbf{Koeffizientenmatrix:}
\[
A =
\quad
\begin{pmatrix}
\textcolor{red}{2} & 3 & \textcolor{blue}{-1} \\
\textcolor{red}{4} & -1 & \textcolor{blue}{5} \\
\textcolor{red}{1} & 2 & \textcolor{blue}{3}
\end{pmatrix}
\quad
, b =
\quad
\textcolor{orange}{
\begin{pmatrix}
5 \\
9 \\
6
\end{pmatrix}
}
\]
Zeilenstuffenform bilden:
\[
\begin{matrix}
2 & 3 & -1 & 5 \\
4 & -1 & 5 & 9 & \vert II - 2 \cdot I \\
1 & 2 & 3 & 6 & \vert 2 \cdot III - I
\end{matrix}
\]
\[
\begin{matrix}
2 & 3 & -1 & 5 \\
0 & -7 & 7 & -1 \\
0 & 1 & 7 & 7 & \vert 7 \cdot III + II
\end{matrix}
\]
\[
\begin{array}{cccc}
\rowcolor{blue!10}
2 & 3 & -1 & 5 \\[5pt]
\rowcolor{red!30}
0 & -7 & 7 & -1 \\[5pt]
\rowcolor{green!20}
0 & 0 & 56 & 48
\end{array}
\]
\textbf{Rückwärtseinsetzen:} \\
\textcolor{green}{
\[
    56x_3 = 48 \vert :56
    \quad \Rightarrow \quad
    x_3 = \frac{48}{56} = \frac{6}{7} 
\]
}
\textcolor{red}{
\[
-7x_2 + 7 \cdot \frac{6}{7} = -1 \quad \Rightarrow \quad -7x_2 + 6 = -1 \quad \Rightarrow \quad -7x_2 = -7 \quad \Rightarrow \quad x_2 = 1
\]
}
\textcolor{blue}{
\[
2x_1 + 3 \cdot 1 - \frac{6}{7} = 5 \quad \Rightarrow \quad 2x_1 + 3 - \frac{6}{7} = 5 \quad \Rightarrow \quad 2x_1 = 5 - 3 + \frac{6}{7} = \frac{32}{7} \quad \Rightarrow \quad x_1 = \frac{16}{7}
\]
}
\textbf{Lösungsmenge:} $\left\{ \left( \frac{16}{7}, 1, \frac{6}{7} \right) \right\}$

\subsection{Beispiel (Unendlich viele Lösungen mit Parameter t)}
\begin{align*}
    x_1 + 2x_2 - x_3 &= 3 \\
    2x_1 + 4x_2 - 2x_3 &= 6 \\
    3x_1 + 6x_2 - 3x_3 &= 9
\end{align*}
\textbf{erweitere Koeffizientenmatrix:}
\[
\begin{pmatrix}
1 & 2 & -1 & 3 \\
2 & 4 & -2 & 6 \\
3 & 6 & -3 & 9
\end{pmatrix}
\]
\textbf{Zeilenstuffenform:}
\[
\begin{matrix}
1 & 2 & -1 & 3 \\
2 & 4 & -2 & 6 & \vert II - 2 \cdot I\\
3 & 6 & -3 & 9 & \vert III - 3 \cdot I
\end{matrix}
\]
\[
\begin{matrix}
1 & 2 & -1 & 3 \\
0 & 0 & 0 & 0 \\
0 & 0 & 0 & 0
\end{matrix}
\]
\textbf{Lösung:} \\
 $x_1 = 3 - 2t$ \\
$x_2 = t$ \\
$x_3 = t$


\subsection{Rang einer Matrix (LGS in Matrixform)}
Der Rang einer Matrix ist die Anzahl der Zeilen, die linear unabhängig sind. \\ Dafür muss die Matrix in Zeilenstuffenform gebracht werden. Der Rang ist dann die Anzahl der Zeilen, die nicht nur aus Nullen bestehen. \\
\textbf{Beispiele:}
\[
A:
rang
\begin{pmatrix}
\textcolor{olive}{1} & 2 & 3 \\
0 & \textcolor{olive}{1} & 2 \\
0 & 0 & 0
\end{pmatrix}
= 2
\quad
,
\quad
B:
rang
\begin{pmatrix}
\textcolor{olive}{1} & 2 & 3 \\
0 & \textcolor{olive}{1} & 2 \\
0 & 0 & \textcolor{olive}{1}
\end{pmatrix}
= 3
\]

\subsection{Defekt einer Matrix (LGS in Matrixform)}
Der Defekt einer Matrix ist die Anzahl von freien Variablen, die in dem LGS der Matrix vorkommen. \\ \\
$n := \text{Anzahl der Spalten}$ \\
$def(A) := n - rang(A)$ \\ \\
Beispiel mit den Matritzen von oben: \\
$def(A) = 3 - 2 = 1$\\
$def(B) = 3 - 3 = 0$

\subsection{Lösbarkeit eines LGS}
\begin{itemize}
    \item $rang(A) <  rang(A, b)$:\quad LGS ist nicht lösbar.
    \item $rang(A) = rang(A, b)$:\quad LGS ist lösbar.
    \item $def(A) = 0$:\quad LGS hat genau eine Lösung.
    \item $def(A) > 0$:\quad LGS hat unendlich viele Lösungen.
\end{itemize}

\section{Matritzen und Vektoren}
\textbf{Videoempfehlung:} \href{https://www.youtube.com/watch?v=fNk_zzaMoSs}{Videoserie 3Blue1Brown - Essence of Linear Algebra 1: Vektoren} \\ \\
Matritzen und Vektoren haben Dimensionen. Die Dimension einer Matrix ist die Anzahl der \textcolor{blue}{Zeilen} und \textcolor{red}{Spalten}. Dies wird Dargestellt als $A \in \mathbb{R}^{\textcolor{blue}{Zeilen}\text{x}\textcolor{red}{Spalten}}$\\ Die Dimension eines Vektors ist die Anzahl der Elemente. \\
\subsection{Transponierte Matrix (\texorpdfstring{$A^T$}{A^T})}
Jede Zeile wird zur Spalte. Die Elemente werden von mit dem Uhrzeigersinn gedreht. \\
\textbf{Beispiel:}
\[
\begin{pmatrix}
1 & \textcolor{blue}{2} & \textcolor{red}{3} \\
4 & \textcolor{blue}{5} & \textcolor{red}{6}
\end{pmatrix}^T
\quad
=
\quad
\begin{pmatrix}
    1 & 4 \\
    \textcolor{blue}{2} & \textcolor{blue}{5} \\
    \textcolor{red}{3} & \textcolor{red}{6}
\end{pmatrix}
\]

\subsection{Addition und Subtraktion}
Matritzen (und Vektoren) können nur addiert oder subtrahiert werden, wenn sie die gleiche Dimension haben. \\
Jedes Element wird mit dem entsprechenden Element der anderen Matrix addiert oder subtrahiert. \\
\textbf{Beispiel:}
\[
\begin{pmatrix}
1 & 2 \\
3 & 4
\end{pmatrix}
+
\begin{pmatrix}
5 & 6 \\
7 & 8
\end{pmatrix}
=
\begin{pmatrix}
6 & 8 \\
10 & 12
\end{pmatrix}
\]

\subsection{Sklarmultiplikation}
Eine Matrix (oder ein Vektor) wird mit einem Skalar multipliziert. \\
Jedes Element der Matrix (oder des Vektors) wird mit dem Skalar multipliziert. \\
\textbf{Beispiel:}
\[
2 \cdot
\begin{pmatrix}
1 & 2 \\
3 & 4
\end{pmatrix}
=
\begin{pmatrix}
2 & 4 \\
6 & 8
\end{pmatrix}
\]

\subsection{Sklarprodukt}
Das Skalarprodukt zweier Vektoren ist die Summe der Produkte der entsprechenden Elemente. \\
\textbf{Beispiel:}
\[
\begin{pmatrix}
\textcolor{blue}{1} \\
\textcolor{green}{2} \\
3
\end{pmatrix}
\cdot
\begin{pmatrix}
\textcolor{blue}{4} \\
\textcolor{green}{5} \\
6
\end{pmatrix}
=
\textcolor{blue}{1 \cdot 4} + \textcolor{green}{2 \cdot 5} + 3 \cdot 6
=
\textcolor{blue}{4} + \textcolor{green}{10} + 18
=
\textcolor{red}{32}
\]

\subsection{Liniearkombination}
Eine Linearkombination ist eine Summe von Skalarmultiplikationen von Vektoren. \\
\textbf{Beispiel:}
\[
2 \cdot
\begin{pmatrix}
1 \\
2
\end{pmatrix}
+
3 \cdot
\begin{pmatrix}
3 \\
4
\end{pmatrix}
=
\begin{pmatrix}
2 \\
4
\end{pmatrix}
+
\begin{pmatrix}
9 \\
12
\end{pmatrix}
=
\begin{pmatrix}
11 \\
16
\end{pmatrix}
\]

\subsection{Analytische Geometrie}
Die Analytische Geometrie löst geometrische Probleme mit algebraischen Methoden. Bietet aber auch die Möglichkeit, algebraische Probleme verständlich zu visualisieren. \\

Vektoren werden als Pfeile dargestellt. Der Anfangspunkt ist der Ursprung. Der Pfeil zeigt in die Richtung des Vektors. \\

\subsubsection{Kartesisches Koordinatensystem}
\begin{itemize}
    \item $n$-dimensionales Koordinatensystem mit $n$ Achsen.
    \item Achsen schneiden sich im Ursprung.
    \item Der Schnittpunkt der Achsen ist der Ursprung.
\end{itemize}
\subsubsection{Addition und Subtraktion}
Zwei Vektoren \textbf{a} und \textbf{b} spannen immer ein Parallelogramm mit den Diagonalen \textbf{$a+b$} und \textbf{$a-b$} auf, so lange sie nicht parallel, ein Vielfaches voneinander oder der Nullvektor sind. \\

\textbf{Beispiel:}
\[
\begin{pmatrix}
1 \\
3
\end{pmatrix}
+
\begin{pmatrix}
8 \\
2
\end{pmatrix}
=
\begin{pmatrix}
9 \\
5
\end{pmatrix}
\]

\[
\begin{pmatrix}
1 \\
3
\end{pmatrix}
-
\begin{pmatrix}
8 \\
2
\end{pmatrix}
=
\begin{pmatrix}
-7 \\
1
\end{pmatrix}
\]

\begin{center}
\begin{tikzpicture}
    % Definition der Achsen
    \begin{axis}[
        axis lines = middle,
        xlabel = {$x$},
        ylabel = {$y$},
        xmin = -10, xmax = 11,
        ymin = -2, ymax = 6,
        axis equal,
        grid = both,
        legend style={at={(1,1)},anchor=north east},
        title = {Vektoraddition und -subtraktion}
    ]

    % Original Vectors
    \addplot[->, blue, thick] coordinates {(0,0) (1,3)};
    \node at (axis cs: 1, 3) [anchor=south] {$\vec{a}$};

    \addplot[->, red, thick] coordinates {(0,0) (8,2)};
    \node at (axis cs: 8, 2) [anchor=south] {$\vec{b}$};

    % Addition and Subtraction of Vectors
    \addplot[->, black, dashed] coordinates {(1,3) (9, 5)};
    \addplot[->, black, dashed] coordinates {(8,2) (9, 5)};
    \addplot[->, green, thick] coordinates {(0,0) (9,5)};
    \node at (axis cs: 9, 5) [anchor=south] {$\vec{a} + \vec{b}$};

    \addplot[->, orange, thick] coordinates {(0,0) (-7,1)};
    \node at (axis cs: -7, 1) [anchor=south] {$\vec{a} - \vec{b}$};

    % Negation of Vector b
    \addplot[->, red, dashed] coordinates {(0,0) (-8,-2)};
    \addplot[->, blue, dashed] coordinates {(-8,-2) (-7,1)};
    \addplot[->, black, dashed] coordinates {(-7,1) (1,3)};

    \end{axis}
\end{tikzpicture}
\end{center}

\subsubsection{Sklarmultiplikation}
Die Skalarmultiplikation eines Vektors \textbf{a} mit einem Skalar $\lambda$ verändert die Länge des Vektors, aber nicht die Richtung. \\
\textbf{Beispiel:}
\[
-1 \cdot
\begin{pmatrix}
1 \\
3
\end{pmatrix}
=
\begin{pmatrix}
-1 \\
-3
\end{pmatrix}
\]

\begin{center}
\begin{tikzpicture}
    % Definition der Achsen
    \begin{axis}[
        axis lines = middle,
        xlabel = {$x$},
        ylabel = {$y$},
        xmin = -2, xmax = 2,
        ymin = -4, ymax = 4,
        axis equal,
        grid = both,
        legend style={at={(1,1)},anchor=north east},
        title = {Sklarmultiplikation}
    ]

    % Original Vector
    \addplot[->, blue, thick] coordinates {(0,0) (1,3)};
    \node at (axis cs: 1, 3) [anchor=south] {$\vec{a}$};

    % Skalarmultiplikation
    \addplot[->, red, thick] coordinates {(0,0) (-1,-3)};
    \node at (axis cs: -1, -3) [anchor=north] {$-1 \cdot \vec{a}$};

    \end{axis}
\end{tikzpicture}
\end{center}

\subsubsection{Zeilenbild LGS}
Die Zeilen eines LGS können als Vektoren interpretiert werden. \\
\textbf{Beispiel:}
\[
\begin{matrix}
2x_1 + 3x_2 &= 5 \\
4x_1 - x_2 &= 9 \\
\end{matrix}
\]
Nach $x_2$ umstellen:
\[
\begin{matrix}
x_2 &= \frac{5}{3} - \frac{2}{3}x_1 \\
x_2 &= 9 + 4x_1
\end{matrix}
\]
\begin{center}
\begin{tikzpicture}
    % Definition der Achsen
    \begin{axis}[
        axis lines = middle,
        xlabel = {$x_1$},
        ylabel = {$x_2$},
        xmin = -2, xmax = 2,
        ymin = -1, ymax = 7,
        axis equal,
        grid = both,
        legend style={at={(1,1)},anchor=north east},
        title = {Zeilenbild LGS}
    ]

    \addplot[thick, blue] {5/3 - 2/3 * x};
    \node at (axis cs: 1, 1) [anchor=south] {\textcolor{blue}{$x_2 = \frac{5}{3}-\frac{2}{3}x_1$}};

    \addplot[thick, red] {9 + 4 * x};
    \node at (axis cs: 1, 5) [anchor=south] {\textcolor{red}{$x_2 = 9 + 4x_1$}};

    \node at (axis cs: -1.6, 2.7) [anchor=east] {$\begin{pmatrix}
    -1.57 \\
    2.7
    \end{pmatrix}$};   

    \end{axis}
\end{tikzpicture}
\end{center}
Der Schnittpunkt der beiden Geraden ist die Lösung des LGS. \\
\textbf{Lösung:} $(x_1, x_2) = (-1.57, 2.7)$

\subsubsection{Spaltenbild LGS}
Die Spalten einer Koeffizientenmatrix eines LGS können als Linearkombination dargestellt werden. \\
\textbf{Beispiel:}
\[
\begin{matrix}
3x_1 - x_2 = 1 \\
x_1 + 2x_2 = 5
\end{matrix}
\Leftrightarrow
x_1 \cdot
\textcolor{red}{
\begin{pmatrix}
3 \\
1
\end{pmatrix}
}
+
x_2 \cdot
\textcolor{blue}{
\begin{pmatrix}
-1 \\
2
\end{pmatrix}
}
=
\textcolor{orange}{
\begin{pmatrix}
1 \\
5
\end{pmatrix}
}
\]

Parallelogramm aufspannen:
\begin{center}
\begin{tikzpicture}

    % Definition der Achsen
    \begin{axis}[
        axis lines = middle,
        xlabel = {$b_1$},
        ylabel = {$b_2$},
        xmin = -2, xmax = 2,
        ymin = -1, ymax = 7,
        axis equal,
        grid = both,
        legend style={at={(1,1)},anchor=north east},
        title = {Spaltenbild LGS}
    ]

    % Original Vectors
    \addplot[->, red, thick] coordinates {(0,0) (3,1)};
    \node at (axis cs: 3, 1) [anchor=south] {\textcolor{red}{$\vec{a}$}};

    \addplot[->, blue, thick] coordinates {(0,0) (-1,2)};
    \node at (axis cs: -1, 2) [anchor=south] {\textcolor{blue}{$\vec{b}$}};

    % Linearkombination
    \addplot[->, orange, thick] coordinates {(0,0) (1,5)};
    \node at (axis cs: 1, 5) [anchor=south] {\textcolor{orange}{$\vec{c}$}};

    \addplot[black, dashed] coordinates {(-1, 2) (-2,4)};
    \addplot[black, dashed] coordinates {(3, 1) (1,5)};
    \addplot[black, dashed] coordinates {(-2,4) (1,5)};

    \node at (axis cs: -1.6, 2.7) [anchor=east] {\textcolor{blue}{$2 \cdot \vec{b}$}};

    \node at (axis cs: -1.25, 5) [anchor=west] {\textcolor{red}{$1 \cdot \vec{a}$}};
    \end{axis}
\end{tikzpicture}
\end{center}
Aus dem Parallelogramm kann die Lösung des LGS abgelesen werden in dem geguckt wird, wie oft der Vektor $\vec{a}$ und $\vec{b}$ addiert werden müssen, um den Vektor $\vec{c}$ zu erreichen. \\
\[
\textcolor{red}{1 \cdot \vec{a}}\textcolor{blue}{ + 2 \cdot \vec{b}} = \textcolor{orange}{\vec{c}}
\]
\textbf{Lösung:} $(x_1, x_2) = (1, 2)$

\subsection{Matrixmultiplikation}
\textbf{\textcolor{red}{Vorraussetzung:}} Die Anzahl der Spalten der ersten Matrix muss der Anzahl der Zeilen der zweiten Matrix entsprechen. \\ \\
\textbf{Achtung!} Die Reihenfolge der Multiplikation ist wichtig. Matrixmultiplikation ist nicht Kummutativ\\ \\
\textcolor{blue}{Die neue Matrix wird so groß wie die Anzahl der Zeilen der ersten Matrix und die Anzahl der Spalten der zweiten Matrix.} \\
\textbf{Beispiel:}
\[
\textcolor{blue}{3}\text{x}\textcolor{red}{4} \cdot \textcolor{red}{4}\text{x}\textcolor{blue}{2} = \textcolor{blue}{3}\text{x}\textcolor{blue}{2}
\]
\subsubsection{Falk-Schema}
Das Falk-Schema ist eine hilfreiche unterstützung bei Matrixmultiplikationen. \\
\[
\begin{matrix}
A \downarrow B \rightarrow  & \begin{pmatrix}
       b_{1,1} & \dots & b_{1,p} \\
         \vdots &  & \vdots \\
         b_{n, 1} & \dots & b_{n, p}
        \end{pmatrix} \\ \\
\begin{pmatrix}
a_{1,1} & \dots & a_{1,m} \\
\vdots &  & \vdots \\
a_{n,1} & \dots & a_{n,m}
\end{pmatrix} & \begin{pmatrix}
c_{1,1} & \dots & c_{1,p} \\
\vdots &  & \vdots \\
c_{n,1} & \dots & c_{n,p}
\end{pmatrix}
\end{matrix}
\]
\newpage

\subsubsection{Zeilen-Spalten-weise}
Jedes Element der $i$-ten Zeile der ersten Matrix wird mit jedem Element der $j$-ten Spalte der zweiten Matrix multipliziert und aufsummiert. (Sklarprodukt) \\
\textbf{Beispiel: $A \cdot B$} \\
\[
\begin{matrix}
    A \downarrow B \rightarrow & \begin{pmatrix}
1 & 2 \\
3 & 4 \\
5 & 6
\end{pmatrix} \\ \\
\begin{pmatrix}
5 & 6 & 7\\
8 & 9 & 1
\end{pmatrix} & \begin{pmatrix}
[5 \cdot 1 + 6 \cdot 3 + 7 \cdot 5] & [5 \cdot 2 + 6 \cdot 4 + 7 \cdot 6] \\
[8 \cdot 1 + 9 \cdot 3 + 1 \cdot 5] & [8 \cdot 2 + 9 \cdot 4 + 1 \cdot 6]
\end{pmatrix}
\end{matrix}
\]
\[
\Rightarrow
\quad
\begin{matrix}
    A \downarrow B \rightarrow & \begin{pmatrix}
1 & 2 \\
3 & 4 \\
5 & 6
\end{pmatrix} \\ \\
\begin{pmatrix}
5 & 6 & 7\\
8 & 9 & 1
\end{pmatrix} & \begin{pmatrix}
    58 & 76 \\
    40 & 58
\end{pmatrix}
\end{matrix}
\]





\end{document}